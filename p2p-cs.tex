
\documentclass[12pt]{amsart}
\usepackage{geometry} % see geometry.pdf on how to lay out the page. There's lots.
\geometry{a4paper} % or letter or a5paper or ... etc
% \geometry{landscape} % rotated page geometry

% See the ``Article customise'' template for come common customisations

\title{Peer to Peer Vs Client-Server : Which is Better?}
\author{Peter Keogh}
\date{}

%%% BEGIN DOCUMENT
\begin{document}

\maketitle
\tableofcontents

\section{Introduction}

The growth of the internet has seen the emergence of two typical network architectures in many applications. These architectures are the Client-Server model and the Peer-to-Peer model. A powerful for the popularity of these architectures is their respective shares of network traffic. HTTP, the protocol of the client server accounts for approximately 46\% of all traffic. P2P comes in at a close second consuming 37\% of total traffic.
The Client-Server model is the driving force behind much innovation today, most obviously in Cloud Computing. Gmail, Facebook and Dropbox are all classic examples of a client-server architecture. It's appeal lies in it's flexibility and robustness. The advent of faster and more reliable internet access in many parts of the world and the growth in the number of users has been a driving factor. People want to be able to access their resources on all of their devices. This "new" model offers robustness. In years gone by users would have to back all their work up on a CD or floppy disk. They would forget to do so at their peril. Now one can work on a project locally and easily save it to a remote server to back it up. 
Using cloud services can allow the user to spend less money on their devices. Thin clients like Google's Chrome book have come to fill a new gap in the market. 
The architecture has also influence the development of client technology. Web browsers are no longer simple programs that display HTML. Fast Javascript engines have been developed like Chrome's V8. Graphics processing has been made possible in the browser by WebGL. The HTML5 specification is being implemented by vendors and as a result browsers have more access to system hardware than ever before.
However, this flexibility comes at a price. Privacy is certainly an issue. While most people consider these services free they do come at a price. It's an internet cliche now that "If You're Not Paying for It; You're the Product". Privacy is a hot topic these days. In light of the revelations of Wikileaks and people like Edward Snowden the public have started asking if the convenience of these services is worth trading their privacy for. 
The client server model has brought with it challenges for law makers. If an Irish customer's data is stored in a server in the USA, what laws govern that data?
Censorship is and freedom of expression are also at play here. The internet provides the public with the power to communicate and organise effectively. The ability to spread information effectively empowers people to overthrow oppressive regimes. However, centralised servers can be easily blocked by tactics such as DNS poisoning. An authoritarian governments control over servers allows it to effectively control it's population's minds.
The Peer-to-peer architecture on the other hand ... Napster, Bittorrent, 


Security
NSA
International
Censorship
Thin clients



\section{The Architectures}
\subsection{Client-Server}

In the Client-Server Architecture users connect over a network to a single server. The clients exchange data with the server while the server provides the clients with services. TOP DOWN, CENTRALISED

\subsection{Peer-to-Peer}
DECENTRALISED, BOTTOM UP, COLLABORATIVE

\subsection{Hybrid Peer-to-Peer}

\end{document}